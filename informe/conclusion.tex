\section{Conclusión}

Un análisis interesante se origina al comparar las diferentes redes, las cuales tienen diferentes tecnologías subyacentes, y cómo esto afecta notoriamente a las fuentes de información estudiadas. Otro factor que afecta a las fuentes es el uso que se le da a la red, por ejemplo como el incremento de la actividad interna de la red aumenta la proporción de paquetes ARP dentro de la muestra, visto en la red empresarial. Sorprende la cantidad de paquetes que el usuario no percibe ya que son para el mantenimiento de la red.

Mediante el cálculo de la entropía podemos determinar que tan predecible es una red, a menor entropía es más fácil predecir. Al analizar los paquetes confirmamos lo que esperábamos: las redes más grandes son menos predecibles, sobre todo cuando tenemos en cuenta que una red grande (y en particular una corporativa) tiene mayor tendencia a incluir interacción entre los hosts de la red.

Otra de las conclusiones de este trabajo es que mientras más grande la red, más \texttt{ARP BROADCAST} podremos percibir, debido a que se requieren mas paquetes de control para mantener el estado de la red en actualizado en cada nodo.

Dejando de lado el análisis de la topología de la red por un momento, resulta interesante comparar la cantidad de información de los símbolos que aparecieron. En la figura 14 podemos notar que, efectivamente, hay símbolos que cargan mucha más información que otros. Si lo que buscamos es identificar nodos distinguidos, nos interesan los símbolos que cargan menor información, pues son los más frecuentes e indican actividad intensa (como puede ser la actividad correspondiente a un router).

Las zonas de colisión también juegan un factor importante en la recolección de datos ya que tendremos otra posibilidad de observar los paquetes que realmente viajan por la red. Este rasgo también se pudo evidenciar en los distintos análisis de las fuentes, al comparar los paquetes unicast contra los broadcast.

Sin embargo, debemos concluir que las fuentes de información modeladas no son las más adecuadas. Las mismas son fuentes de memoria nula cuando en realidad sabemos que, por ejemplo, captar la circulación de un paquete \texttt{IS-AT} es más probable si ya captamos un \texttt{WHO-HAS}. Sería interesante analizar cuales serian los resultados si tuviésemos en cuenta paquetes anteriores para determinar la probabilidad de uno nuevo.

Otras herramientas estadísticas como la moda de la muestra podrían ser útiles para encontrar símbolos distinguidos, sobre todo para la segunda fuente.

