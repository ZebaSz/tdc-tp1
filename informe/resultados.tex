\section{Resultados}

\subsection{Modelo de fuente \texorpdfstring{$S_1$}{S1}}

A continuación se dispone el análisis de los resultados obtenidos modelando las redes como fuentes de información con memoria nula.

\subsubsection{Red corporativa}

Los siguientes graficos muestran para cada red la información para cada uno de sus símbolos capturados, las probabilidades de que estos sean registrados, la entropía registrada y la entropía máxima.

\begin{figure}[H]
	\begin{minipage}{0.49\textwidth}
		\centering
		\includegraphics[width=\linewidth]{imagenes/despegar_barras_prob}
		\caption{Red corporativa - Probabilidad}
		\label{despe_barras_prob}
	\end{minipage}
	\begin{minipage}{0.49\textwidth}
		\centering
		\includegraphics[width=\linewidth]{imagenes/despegar_barras_info}
		\caption{Red corporativa - Información}
		\label{despe_barras_info}
	\end{minipage}
\end{figure}

Sobre esta red observamos un trafico medianamente distribuido de los símbolos obtenidos. La mitad de ellos son \texttt{IPV4 UNICAST}, casi un 20 \% de \texttt{IPV4 BROADCAST} y entre 12\% y 8\% el resto. Puede verse una gran cantidad de informacion proveniente del símbolo \texttt{LLDP UNICAST}. Este protocolo es conocido cómo \textit{Link Layer Discovery Protocol}. La prescencia de este tipo de paquete se explica con que se usa como un componente en aplicaciones de administración y monitoreo de red, lo cual es común en redes corporativas. La baja frecuencia de este tipo de paquete repercute en que la información del símbolo sea muy alta.

Con el fin de realizar un análisis más exhaustivo, resulta útil estudiar los porcentajes relativos de aparición de los distintos símbolos de la fuente S1 para cada red.

\begin{figure}[H]
	\begin{minipage}{0.49\textwidth}
		\centering
		\includegraphics[width=\linewidth]{imagenes/despegar_torta_simbolos}
		\caption{Red corporativa - Símbolos}
		\label{despe_torta_simb}
	\end{minipage}
	\begin{minipage}{0.49\textwidth}
		\centering
		\includegraphics[width=\linewidth]{imagenes/despegar_torta_tipos}
		\caption{Red corporativa - Tipos}
		\label{despe_torta_tipos}
	\end{minipage}
\end{figure}

En este caso, existe una desproporción entre paquetes broadcast y unicast. Sin embargo, la proporción de paquetes broadcast y además la cantidad de paquetes ARP parece indicar que existe bastante interacción entre los nodos dentro de la red. Al ser una red con gran cantidad de hosts donde varios de ellos se comunican entre sí, estos necesitan mantenerse actualizados en cuanto a la información sobre la red. Es interesante observar la cantidad de paquetes tipo \texttt{IPV4 Broadcast}, este tipo de paquete es mayormente utilizado por protocolos de capas superiores, como NetBT, para tener más una mayor información sobre la red, ya que en un ambiente corporativo es muy necesario para detectar errores rápidamente.

\subsubsection{Red pública (McDonald's)}

\begin{figure}[H]
	\begin{minipage}{0.49\textwidth}
		\centering
		\includegraphics[width=\linewidth]{imagenes/mac_barras_prob}
		\caption{Red pública(McDonald's) - Probabilidad}
		\label{mac_barras_prob}		
	\end{minipage}
	\begin{minipage}{0.49\textwidth}
		\centering
		\includegraphics[width=\linewidth]{imagenes/mac_barras_info}
		\caption{Red pública(McDonald's) - Información}
		\label{mac_barras_info}		
	\end{minipage}
\end{figure}

La principal motivación para estudiar la red de McDonald's fue poder observar un medio con mucha entrada y salida de dispositivos, el hecho de que se tratara de una red inalámbrica permitió además contrastar las diferencias frente a la primer captura sobre una red cableada. Se nota la gran cantidad de paquetes \texttt{ARP BROADCAST}, son el 60\% de los capturados por lo que la información que proveen es escasa. Nuestra hipótesis es que la continua fluctuación de hosts en esta red provoca que se disparen muchos paquetes de control para mantener actualizado el estado de la misma. 

\begin{figure}[H]
	\begin{minipage}{0.49\textwidth}
		\centering
		\includegraphics[width=\linewidth]{imagenes/mac_torta_simbolos}
		\caption{Red pública(McDonald's) - Símbolos}
		\label{mac_torta_simb}
	\end{minipage}
	\begin{minipage}{0.49\textwidth}
		\centering
		\includegraphics[width=\linewidth]{imagenes/mac_torta_tipos}
		\caption{Red pública(McDonald's) - Tipos}
		\label{mac_torta_tipos}
	\end{minipage}
\end{figure}

En la captura realizada sobre la red pública podemos ver que la cantidad de paquetes destinados al control de la red es considerablemente mayor con respecto a las red corporativa y la doméstica.

\subsubsection{Red doméstica}

\begin{figure}[H]
	\begin{minipage}{0.49\textwidth}
		\centering
		\includegraphics[width=\linewidth]{imagenes/manu_casa_barras_prob}
		\caption{Red doméstica - Probabilidad}
		\label{casa_barras_prob}
	\end{minipage}
	\begin{minipage}{0.49\textwidth}
		\centering
		\includegraphics[width=\linewidth]{imagenes/manu_casa_barras_info}
		\caption{Red doméstica - Información}
		\label{casa_barras_info}
	\end{minipage}
\end{figure}

Para el grafico de la red doméstica podemos observar un predominancia en los paquetes \texttt{IPV4 UNICAST}, conformando mas de un 80\% de los paquetes que transitan por la red, seguido por un 18\% de paquetes \texttt{IPV6 UNICAST}. La entropía es de 0.797. Este número resulta mucho mas baja que las demás, y creemos que se debe justamente a que la gran mayoría de los paquetes es \texttt{UNICAST}, y esto provoca que el tráfico de la red sea fácil de predecir. La entropía es mucho menor en esta red que en las dos anteriores, lo cual indica una peor distribución en cuanto a los símbolos analizados. De la misma forma podemos observar que la red corporativa presenta una mejor distribución de los símbolos, lo cual se condice con su entropía, siendo esta la mayor de entre las tres redes.

\begin{figure}[H]
	\begin{minipage}{0.49\textwidth}
		\centering
		\includegraphics[width=\linewidth]{imagenes/manu_casa_torta_simbolos}
		\caption{Red doméstica - Símbolos}
		\label{casa_torta_simb}
	\end{minipage}
	\begin{minipage}{0.49\textwidth}
		\centering
		\includegraphics[width=\linewidth]{imagenes/manu_casa_torta_tipos}
		\caption{Red doméstica - Tipos}
		\label{casa_torta_tipos}
	\end{minipage}
\end{figure}

Como puede observarse en los gráficos, los resultados obtenidos para la fuente $S_1$ muestran que los paquetes de tipo broadcast son casi inexistentes. Este fenómeno se puede explicar, por el siguiente motivo: los paquetes de tipo broadcast suelen corresponder con protocolos de control, como ARP, y dado que las comunicaciones en general se dan exclusivamente entre el default gateway y los demás nodos, los protocolos de control no son requeridos.

\subsection{Modelo de fuente \texorpdfstring{$S_2$}{S2}}

Antes que nada, cabe destacar los cambios en la cantidad de símbolos con el modelo anterior. Este hace una distinción por cada host distinto dentro de las capturas, y al tener capturas en las que hay muchos dispositivos en juego, la cantidad de símbolos distintos aumenta linealmente con ellos.

\subsubsection{Red corporativa}

Dada la naturaleza de la red corporativa, es de esperarse que la cantidad de símbolos en $S_2$ sea elevada, ya que los mismos se distinguen por IP. Bajo nuestra categorización, encontramos un total de 239 símbolos.

\begin{figure}[H]
	\centering
	\includegraphics[width=\linewidth]{imagenes/despegar_barras_all}
	\caption{Red corporativa - Todos los símbolos}
\end{figure}

Pudimos observar que si bien hay más paquetes de tipo \texttt{WHO-HAS}, los paquete \texttt{IS-AT} no son pocos, casi el 25\% de los paquetes capturados, esto nos habla de una gran interacción entre los dispositivos pertenecientes a la red. En primera instancia, esperabamos encontrar una proporción cercana a 1:1, siendo que para cada pedido de \texttt{WHO-HAS} suele haber un nodo que lo responda. Nuestra hipótesis al respecto es que no tenemos acceso a todas las respuestas \texttt{IS-AT}, o también que esas redes no existen (pero esto es muy poco probable). Otra posible explicación es que la red está switcheada, lo cual justamente limita la cantidad de respuestas \texttt{IS-AT} que podemos ver porque limita la zona de colisión.

Un detalle que nos interesó en particular tiene que ver con los hosts que proveen menor informacion. Para ver esto, usamos un gráfico una pequeña porción de los hosts:

\begin{figure}[H]
	\centering
	\includegraphics[width=.5\linewidth]{imagenes/despegar_hosts}
	\caption{Red corporativa}
\end{figure}

Hay 6 símbolos de tipo \texttt{WHO-HAS} por debajo de la entropía. Tanto estos nodos como algunos de los otros presentes en el gráfico se pueden considerar distinguidos, pues posiblemente sean las IPs referidas a gateways o a servidores que deben ser recurridos frecuentemente por distintos usuarios. De todos modos, resulta dificil inferir a partir de estos datos cuales se tratan de routers o gateways. En particular, utilizando otras herramientas daría la impresión que la red utiliza IPs terminadas en \texttt{.254} (es decir, de la forma \texttt{10.x.x.254}) para los gateways, y sin embargo el símbolo de mayor frecuenta es \texttt{WHO-HAS 10.254.97.194} que no corresponde con ese patrón.

\subsubsection{Red pública}


\begin{figure}[H]
	\centering
	\includegraphics[width=.5\linewidth]{imagenes/mac_hosts}
	\caption{Red pública}
\end{figure}

Podemos observar que la cantidad de símbolos decreció abruptamente con respecto a la red anterior, y esto evidencia la diferencia de hosts que existe entre una red y otra. Era esperable que la cantidad de hosts sea mucho menor a la corporativa puesto que la primer red tiene muchos dispositivos destinados a la empresa mientras que la pública solo cuenta con algunos esporádicos clientes que deciden conectarse (por más que hayan muchos consumidores en el lugar, eso no significa que se conecten a la red).

Nótese que en el grafico la información del símbolo \texttt{WHO-HAS 172.19.0.1} se ubica por debajo de la entropía, es el que menos información aporta. Por ello deducimos que al ser un símbolo de \texttt{WHO-HAS}, 172.19.0.1 es el router, ya que es el más requerido. Podemos concluir esto último ya que cuanta menos información aporte mayor es su probabilidad de aparición.

Otro detalle a tener en cuenta es nuestra posición en la red: al tratarse de una conexión inalámbrica, es mucho menos probable que captemos paquetes que no nos corresponden.

Cabe destacar que al realizar el análisis de esta red encontramos una cantidad bastante alta de paquetes que consideramos \textit{gratuitous} (mismo destino \texttt{IS-AT} y origen \texttt{WHO-HAS}. Al analizarlos a fondo encontramos un protocolo de descubrimiento de dispositivos (SSDP). Con este protocolo, después de encontrar un paquete de tipo \texttt{discover}, los dispositivos envian un paquete ARP \textit{gratuitous}, que genera ruido e información que no necesitamos en el analisis.

Nótese que en el grafico la información del símbolo \texttt{WHO-HAS 172.19.0.1} se ubica por debajo de la entropía, es el que menos información aporta. Por ello deducimos que al ser un símbolo de \texttt{WHO-HAS}, 172.19.0.1 es el router, ya que es el más requerido. Podemos concluir esto último ya que cuanta menos información aporte mayor es su probabilidad de aparición.

Por último, al tratarse de una red pública es de esperar que haya una concentración fuerte de paquetes, siendo \texttt{WHO-HAS 172.19.0.1} el símbolo más común. Dado que los equipos de esta red están mayormente aislados y no se comunican entre sí, el tráfico de control es extremadamente predecible.

En ese sentido, a nivel topológico, podemos decir que el router es el único nodo distinguido. El otro paquete de baja información es un \texttt{IS-AT}, que podemos inferir proviene de nuestro propio equipo ya que es el único que podemos escuchar.

\subsubsection{Red doméstica}

\begin{figure}[H]
	\centering
	\includegraphics[width=.5\linewidth]{imagenes/manu_casa_hosts}
	\caption{Red doméstica}
\end{figure}

Como era de esperar, la red doméstica contiene muy pocos hosts y es altamente predecible: los nodos se comunican poco entre sí, y la mayor parte del tiempo lo hacen el nodo principal y distinguido, el router. El mismo se encuentra posicionado como default gateway en una IP muy común en redes hogareñas: \texttt{192.168.0.1}.

No obstante, el tráfico resulta menos predecible que el de la red pública. Esto nos sorprendió bastante, aunque es bastante lógico por un detalle previamente mencionado: la red doméstica es de confianza e incluye un cierto grado de interacción entre los distintos hosts. En contraste, la red pública mantiene a los hosts mayormente aislados entre sí, ya que el único propósito de la misma es el acceso a Internet.

La distribución de los símbolos también confirma nuestras ideas sobre la posición de nuestro dispositivo en la red: la gran mayoría de los paquetes capturados son \texttt{WHO-HAS}, un broadcast, mientras que los \texttt{IS-AT} obtenidos son muy esporádicos. Considerando que todos los dispositivos se comunican exclusivamente a través del router, sería lógico que el mismo solo nos relegue los \texttt{IS-AT} correspondientes a él mismo, además de aquellos que son enviados por nuestras computadoras.

Nos resultó llamativa la aparición de algunos símbolos que, bajo nuestro análisis, no deberían figurar en una red doméstica. Se tratan de aquellos cuyas IPs no están dentro del rango de una red privada hogareña típica (\texttt{192.168.0.255/24}). Una de ellas, \texttt{100.102.216.138}, se encuentra en un rango reservado a NAT a nivel proveedor. Sin embargo, no tenemos explicación para las demás IPs.