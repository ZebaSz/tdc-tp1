\section{Resultados}

A continuación se dispone el análisis de los resultados obtenidos modelando las redes como fuentes de información con memoria nula. Cada símbolo está formado por la combinación entre el tipo de destino de la trama (\texttt{UNICAST} o \texttt{BROADCAST}) y el protocolo de la capa inmediata superior encapsulado en la misma. \\

$<$ protocolo, tipo destino $>$ \\


Los siguientes graficos muestran para cada red la información para cada uno de sus símbolos capturados, las probabilidades de que estos sean registrados, la entropía registrada y la entropía máxima.

GRAFICO DESPEGAR

En cuanto a la red laboral observamos un trafico medianamente distriubido de los símbolos obtenidos. La mitad de ellos son \texttt{IPV4 UNICAST}, casi un 20 \% de \texttt{IPV4 BROADCAST} y entre 12\% y 8\% el resto. Puede verse una gran cantidad de informacion proveniente del símbolo \texttt{LLDP UNICAST}. Este protocolo es conocido cómo \textit{Link Layer Discovery Protocol}. La prescencia de este tipo de paquete se explica con que  se usa como un componente en aplicaciones de administración y monitoreo de red, lo cual es común en redes corporativas.

La entropía es de 1.949. 

GRAFICO RED DOMESTICA

Para el grafico de la red doméstica podemos observar un predominancia en los paquetes \texttt{IPV4 UNICAST}, conformando mas de un 80\% de los paquetes que transitan por la red, seguido por un 18\% de paquetes \texttt{IPV6 UNICAST}. La entropía es de 0.797. Este número resulta mucho mas baja que las demás, y creemos que se debe justamente a que la gran mayoría de los paquetes es \texttt{UNICAST}, y esto provoca que el tráfico de la red sea fácil de predecir.



Una posible hipótesis de por que estas redes presentan entropía mayor es, quizás, que el gran tamanño de estas y los posibles tipos de interacciones entre los dispositivos conectados a las redes resulten en una mayor proporción del trafico destinado al control.



Con el fin de realizar un an ́alisis m ́as exhaustivo, resulta u ́til estudiar los porcentajes relativos de aparici ́on de los distintos s ́ımbolos de la fuente S1 para cada red.